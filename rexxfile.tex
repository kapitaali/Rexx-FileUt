\documentclass{article}

% if run through pdftex, use ps fonts and turn on hyperlinking
\ifx\pdfoutput\undefined
 \def\rarrow{\ensuremath{\to}}
\else
 \usepackage[pdftex,pdfborder=0 0 0]{hyperref}

 \usepackage{times}
 \font\symbol=psyr
 \def\rarrow{{\symbol\char174}}
 \pdfinfo { /Title (Rexx File Utility Functions)
  /Author (Patrick TJ McPhee)
  /Subject (W32funcs User Guide)
  /Keywords (rexx,process control)
 }

\fi

\def\_{\textunderscore\penalty\hyphenpenalty}
\advance\textwidth by 1 in
\advance\textheight by 1 in
\advance\oddsidemargin by -.5in
\advance\topmargin by -.5in
\advance\labelwidth by 4 in

\begin{document}
\let\thepage\relax

\title{Rexx File Utility Functions}
\author{Patrick TJ McPhee (ptjm@interlog.com)\\
        DataMirror Corporation}
\date{20 April 2002}

\maketitle

\eject
\setcounter{page}{1}
\pagenumbering{roman}

\tableofcontents

\vfil\eject
\setcounter{page}{1}
\pagenumbering{arabic}

\section{Introduction}

This package defines a set of functions which allow interaction with
running processes using pipes.
The idea is to have some special kinds of files using an interface much
like linein and lineout. This could actually be done using system-dependent
parameters to the stream() function, however there is no portable way
to extend the stream() interface for all possible interpreters.
In the future, I plan to provide support for handling files compressed
with gzip, and the tar and zip archive formats, which will be accessible
through the same interface.

Although the library is ultimately meant to support a variety of special
file formats, it isn't there yet, and I'm in a rush, so this
documentation will mostly talk about reading and writing to processes.

The interface is modelled on the standard Rexx I/O functions. The library
provides an open and a close routine, however it is not necessary to
`open' a file before reading from or writing to it.
The open function exists primarily to allow two processes which have
exactly the same command-line to run concurrently. Processes which are
started using the read and write functions are identified by their
command line, so the first write to `sort -n' will start a new process,
and the second write will go to the same process. The open function
returns a handle which is independent of the command being run, which
would allow two sorts to go on concurrently.

\section{Function Descriptions}


\subsection{List of Routines}

\begin{enumerate}

\item[FileOpen](file,how) $\rightarrow$ handle: starts a command
running, and returns a handle;
\item[FileClose](file[,stream]) $\rightarrow$ 0 or failure: either
closes all open streams to a process and waits for the process to exit,
or closes the specified stream;
\item[FileLineIn](file[,[line][,[count]][,stream]]) $\rightarrow$ data:
reads a line from either the standard output or standard error stream of
a process;
\item[FileCharIn](file[,[line][,[count]][,stream]]) $\rightarrow$ data:
reads {\it count} characters from either the standard output or standard error stream of
a process;
\item[FileLineOut](file[,[string][,line]]) $\rightarrow$ 0 or count:
writes a string and a new-line to the input stream of a process;
\item[FileCharOut](file[,[string][,line]]) $\rightarrow$ 0 or count:
writes a string to the input stream of a process;
\item[FileLines](file[, stream]) $\rightarrow$ 0 or 1: tells whether
there's anything left to be read from a stream;
\item[FileChars](file[, stream]) $\rightarrow$ 0 or 1: tells whether
there's anything left to be read from a stream.

\end{enumerate}


\subsection{FileOpen}

\begin{verbatim}
handle = FileOpen(file,how)
\end{verbatim}

Open a file using method {\it how}.  Currently, {\it how} must be `pipe'
and {\it file} is the command to execute.
{\it handle} is a 10-digit number which should be passed in the {\it
file} argument to the other functions.

You don't have to call FileOpen() for most operations.
If you don't call FileOpen(), and pass the file name to the other functions,
they will open the file if it is necessary and it makes sense to do so.
The only time fileopen()
is needed is if the exact same file needs to be opened more than
one time concurrently. In this case, {\it  handle} differentiates between
the two files. When open methods other than `pipe' are available,
FileOpen() will be required to open files which are not of the default
type (there may be a new function to change the default type from `pipe'
to whatever's most convenient for the program at hand. We'll see.)

For the `pipe' open method, {\it file} is the command to execute. Under NT,
any string which can be passed to CreateProcess() is acceptable. So far
as I know, this means spaces delmit arguments, and double-quotes can be used
to create arguments with spaces in them.
\verb|^| can be used to escape quotes.

Under Unix, the library parses
the command using essentially the same rules as the Bourne shell.
Spaces delimit arguments, and either single- or double-quotes can be used to
create arguments with spaces in them. \verb|\| is the escape character,
and it escapes either spaces or quotes. Within single-quotes, there is no
escape character. Thus \verb|sort file\ with\ spaces|,
\verb|sort "file with spaces"|, and \verb|sort 'file with spaces'| all
invoke sort with the single argument `file with spaces'. However,
\verb|sort "file with \"quote\""|
refers to a file
\verb|file with "quote"| while
\verb|sort 'file with \"quote\"'| refers to a file
\verb|file with \"quote\"|.
Don't be too clever, and you can't go wrong.

\subsection{FileClose}

\begin{verbatim}
rc = FileClose(file[,stream])
\end{verbatim}

If {\it stream} is not specified, FileClose() closes all streams to the
process specified by {\it file} then waits for the process to exit.
{\it file} can be a command name or a handle returned by FileOpen(). 
In this case, FileClose() returns the process return code.

If {\it stream} is specified, it should be `in', `out', or `error', and
it closes the corresponding stream. Note that `in' refers to the
process's input stream, which is what the rexx program might have been
writing to. This seems a bit counter-intuitive at first, but I think
it makes sense. Similarly, `out' refers to the process's standard output
stream, and `error' refers to its standard error stream. Many filters
won't start writing until you've fed them an awful lot of data, or
you've closed their input stream.

\subsection{FileLineIn}

\begin{verbatim}
data = FileLineIn(file[,[line][,[count]][,stream]])
\end{verbatim}

Reads a line from either the standard output or standard error stream of
a process. {\it file} can be either a command or a handle returned by
FileHandle(). See FileOpen() for information about the command format.

If {\it line} is specified, the running process is ended, and a new
process is started. The following
\begin{verbatim}
 cmd = 'sort -n'
 call FileClose cmd
 data = FileLineIn(cmd)
\end{verbatim}
is equivalent to
\begin{verbatim}
 cmd = 'sort -n'
 data = FileLineIn(cmd, 1)
\end{verbatim}

If {\it count} is specified, it should be 0 or 1. 1 means to read a
line, 0 means to evaluate the function for side-effects (starting or
re-starting the command).

{\it stream} can be `out' to read from the standard output, or `error'
to read from standard error. The default is `out'.

If there are no characters to read but the stream is still open,
FileLineIn() will wait until characters are available. If you try to
read and write to and from the same running process, you can deadlock
yourself ({\it i.e.}, you can be waiting for the other process to write
something at the same time it's waiting for you to write something).

On error, FileLinIn() returns an empty string.

\subsection{FileCharIn}

\begin{verbatim}
data = FileCharIn(file[,[line][,[count]][,stream]])
\end{verbatim}

This is just like FileLineIn(), except {\it count} is the number of
characters to read.

\subsection{FileLineOut}

\begin{verbatim}
count = FileLineOut(file[,[string][,line]])
\end{verbatim}

Writes a string and a new-line to the input stream of a process.
{\it file} can be the command to run or a handle returned by FileOpen().
{\it string} is the string to write.

If {\it line} is specified, the process is re-started. See FileLineIn()
to see what I mean by that.

If {\it string} is not specified, the function is evaluated for its
side-effects (starting or re-starting the process).

\subsection{FileCharOut}

\begin{verbatim}
FileCharOut(file[,[string][,line]])
\end{verbatim}

FileCharOut() is just the same as FileLineOut() except a new-line is not
written to the output stream.

\subsection{FileLines}

\begin{verbatim}
rc = FileLines(file[,stream])
\end{verbatim}

Returns 0 if there is nothing to be read from the input stream, or 1 if
there is something to be read. {\it file} can be a command name or
a handle returned by FileOpen(). {\it stream} can be `out' for standard
output (the default) or `error' for standard error.

\subsection{FileChars}

\begin{verbatim}
rc = FileChars(file[,stream])
\end{verbatim}

FileChars() is exactly the same as FileLines().

\end{document}
